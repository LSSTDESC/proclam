\section{Introduction}
\label{sec:intro}

\aim{Add motivation with background on LSST and other surveys.}

\aim{Write background on \plasticc, Kaggle, previous classification challenges.}

\begin{itemize}
\item    The metric must return a single scalar value.
\item    The metric must be well-defined for non-binary classes.
\item    The metric must balance diverse science use cases in the presence of heavily nonuniform class prevalence.
\item    The metric must respect the information content of probabilistic classifications.
\item    The metric must be able to evaluate deterministic classifications.
\item    The metric must be interpretable, meaning it gives a more optimal value for "good" mock classifiers and a less optimal value for mock classifiers plagued by anticipated systematic errors; in other words, it must pass basic tests of intuition.
\item    The metric must be reliable, giving consistent results for different instantiations of the same test case.
\end{itemize}

The Probabilistic Classification Metric (ProClaM) code used in this exploration of performance metrics is publicly available on GitHub.\footnote{\url{https://github.com/aimalz/proclam}}
