% 
% ======================================================================
\RequirePackage{docswitch}
% \flag is set by the user, through the makefile:
%    make note
%    make apj
% etc.
\setjournal{\flag}

\documentclass[\docopts]{\docclass}

% You could also define the document class directly
%\documentclass[]{emulateapj}

% Custom commands from LSST DESC, see texmf/styles/lsstdesc_macros.sty
\usepackage{lsstdesc_macros}

\usepackage{graphicx}
\graphicspath{{./}{./figures/}}
\bibliographystyle{apj}

% Add your own macros here:



% 
% ======================================================================

\begin{document}

\title{ {{ cookiecutter.title }} }

\maketitlepre

\begin{abstract}

{{ cookiecutter.description }}

\end{abstract}

% Keywords are ignored in the LSST DESC Note style:
\dockeys{latex: templates, papers: awesome}

\maketitlepost

% ----------------------------------------------------------------------
% 

\section{Introduction}
\label{sec:intro}

This is a paper and note template for the LSST DESC \citep{Overview,ScienceBook,WhitePaper}.
You can delete all this tutorial text whenever you like.

You can easily switch between various \LaTeX\xspace styles for internal notes and peer reviewed journals.
Documents can be compiled using the provided \code{Makefile}.
The command \code{make} with no arguments compiles \code{main.tex} using the  \code{lsstdescnote.cls} style.
If you want to upgrade your Note into a journal article, just choose a journal name, between \code{make apj} (ApJ preprint format), \code{make apjl} (which uses the \code{emulateapj} style), \code{make prd}, \code{make prl}, and \code{make mnras}.


% ----------------------------------------------------------------------

\section{Commands}
\label{sec:commands}

There are a number of useful \LaTeX\xspace commands predefined in \code{macros.tex}.
Notice that the section labels are prefixed with \code{sec:} to allow the use of the \verb=\secref= command to reference a section (\ie, \secref{intro}).
Figures can be referenced with the \verb=\figref= command, which assumes that the figure label is prefixed with \code{fig:}.
In \figref{example} we show an example figure.
You'll notice that the actual figure file is found in the \code{figures} directory.
However, because we have specified this directory in our \verb=\graphicspath= we do not need to explicitly specify the path to the image.

The \code{macros.tex} package also contains some conventional scientific units like \angstrom, \GeV, \Msun, etc. and some editorial tools for highlighting \FIXME{issues}, \CHECK{text to be checked}, \COMMENT{comments}, and \NEW{new additions}.


% ----------------------------------------------------------------------

\section{Methods}
\label{sec:methods}

Similar to the figure before, here we have included a table of data from \code{tables/table.tex}.
Notice that again we are able to reference \tabref{example} with the \verb=\tabref= command using the \code{tab:} prefix.
Also notice that we haven't needed to specify the full path to the table because in the \code{Makefile} we include \code{./tables} directory in the \code{\$TEXINPUTS} environment variable.

\begin{table}
  \begin{center}
  \caption{Example table. \label{tab:example}}
  %\begin{ruledtabular}
  \begin{tabular}{lccc}
\hline\hline
Column 1 & Column 2 & Column 3 &  Column 4 \\[3pt]  
     &    $\deg$     & $\kpc$   &  $\deg$ \\[4pt]
\hline
Obj1 & (0,0) & 10 & 0.1 \\
... & ... & ... & ... \\
ObjN & (0,0) & 10 & 0.1
\\\hline\hline
\end{tabular}
\end{center}
%\end{ruledtabular}
\end{table}

%\begin{\tabletype}{l ccccccc }
%\tablewidth{0pt}
%\tabletypesize{\tiny}
%\tablecaption{ An example table. \label{tab:example}}
%\tablehead{
%(1) & (2) & (3) & (4) & (5) & (6) & (7) & (8)\\
%Name & GLON,GLAT & Distance & $r_{1/2}$ & $\log_{10}(J_{\rm meas})$ & $\log_{10}(J_{\rm pred})$ & Sample & Refrence \\
% & (deg) & (kpc) & (pc) & $\log_{10}(\GeV^2 \cm^{-5})$ & $\log_{10}(\GeV^2 \cm^{-5})$ & & 
%}
%\startdata
%Bootes I                     & 358.08,69.62   & 66  & 189  & $18.8 \pm 0.2$ & 18.5           & I,S,C & ... \\
%\\
%...\\
%\\
%Willman 1                    & 158.58,56.78   & 38  & 19   & $19.1 \pm 0.3$ & 18.9           & I,S & ... \\
%\enddata
%{\footnotesize \tablecomments{ (1) The first column. (2) The second column ...}}
%\end{\tabletype}


Equations appear as follows, and can be referred to as, for example, \eqnref{example} -- just as for tables, we use the \verb=\eqnref= command using the \code{eqn:} prefix.
\begin{equation}
  \label{eqn:example}
  \langle f(k) \rangle = \frac{ \sum_{t=0}^{N}f(t,k) }{N}
\end{equation}


% ----------------------------------------------------------------------

\section{Results}
\label{sec:results}

\figref{example} shows an example figure, referred to with the \verb=\figref= command and the \code{fig:} prefix.

\begin{figure}
\includegraphics[width=0.9\columnwidth]{example.png}
\caption{An example figure: the LSST DESC logo, copied from \code{texmf/logos/desc-logo.png} into \code{figures/example.png}. \label{fig:example}}
\end{figure}


% ----------------------------------------------------------------------

\section{Discussion}
\label{sec:discussion}

If you are planning on committing your paper to GitHub, it's a good idea to write your tex as one sentence per line.
This allows for an easier \code{diff} of changes.
It also makes sense to think of latex as \emph{code}, and sentences as logical statements, occupying one line each.
Each line must ``compile'' in the mind of the reader.


% ----------------------------------------------------------------------

\section{Conclusions}
\label{sec:conclusions}

Here's a summary of what we just reported.

We can draw the following well-organized and neatly-formatted conclusions:
\begin{itemize}
  \item This is important.
  \item We can measure some number with some precision.
  \item This has some implications.
\end{itemize}

Here are some parting thoughts.


% ----------------------------------------------------------------------

\subsection*{Acknowledgments}

Here is where you should add your specific acknowledgments, remembering that some standard thanks will be added via the \code{acknowledgments.tex} and \code{contributions.tex} files.

% 
This is the text imported from \code{acknowledgments.tex}, and will be replaced by some standard LSST DESC boilerplate at some point.
% 


Author contributions are listed below. \\
A.I.~Malz: conceptualization, data curation, formal analysis, investigation, methodology, project administration, software, supervision, validation, visualization, writing - original draft \\
R.~Hlozek: data curation, formal analysis, funding acquisition, investigation, project administration, software, supervision, validation, visualization, writing - original draft \\
T.~Alam: investigation, software, validation \\
A.~Bahmanyar: formal analysis, investigation, methodology, software, writing - original draft \\
R.~Biswas: conceptualization, methodology, software \\
E.E.O.~Ishida: conceptualization, project administration, supervision \\
G.~Narayan: data curation, formal analysis \\
D.~Jones: software \\
A.~Mahabal: data curation, software \\
R.~Martinez-Galarza: data curation, software, visualization \\
C.~Setzer: software \\


%{\it Facilities:} \facility{LSST}

% Include both collaboration papers and external citations:
\bibliography{lsstdesc,main}

\end{document}
% ======================================================================
% 
