\section{Methods}
\label{sec:methods}

%\begin{itemize}
%\item    The metric must return a single scalar value.
%\item    The metric must be well-defined for non-binary classes.
%\item    The metric must balance diverse science use cases in the presence of heavily nonuniform class prevalence.
%\item    The metric must respect the information content of probabilistic classifications.
%\item    The metric must be able to evaluate deterministic classifications.
%\item    The metric must be interpretable, meaning it gives a more optimal value for "good" mock classifiers and a less optimal value for mock classifiers plagued by anticipated systematic errors; in other words, it must pass basic tests of intuition.
%\item    The metric must be reliable, giving consistent results for different instantiations of the same test case.
%\end{itemize}

To optimally discriminate between classification techniques, there must be a performance metric, a single scalar value quantifying how appropriate a classifier is for the task at hand.
Choosing a metric for \plasticc\ therefore is logically entwined with the challenge goals.
The goal of \snphotcc\ was to identify good classifiers of photometrically identified SN Ia used for cosmology analysis.
The classifiers were ranked using a metric that optimizes the purity of any resultant data set.
\plasticc, on the other hand, differs from \snphotcc\ in that its goals are not tied to one application or even one type of transient or variable class.
As such, the choice of a metric is not so simple for \plasticc's diverse science goals.

%he posterior probabilities of class must be accurate for use in inference, and they must be evaluated in a way that does not excessively favor any one science application.

In Section~\ref{sec:deterministic}, we review familiar classification metrics in astronomy.
In Section~\ref{sec:probabilistic}, we introduce metrics appropriate for probabilistic classification.
We take weighted averages of the per-object metrics with per-class weights described in Section~\ref{sec:weights}.

\subsection{Deterministic metrics}
\label{sec:deterministic}

We begin with a presentation of a subset of classification metrics that have been used in the evaluation of astronomical light curve classifiers in the recent past.
The metrics we highlight make use of the notions of true positive, false positive, and false negative counts from binary deterministic classification.
We briefly define the \textit{efficiency} $\epsilon \equiv \mathrm{TP} / (\mathrm{TP} + \mathrm{FN})$ and \textit{purity} $\pi \equiv \mathrm{TP} / (\mathrm{TP} + \mathrm{FP})$.

\subsubsection{Science-motivated metrics}
\label{sec:science}

The goal of the \snphotcc\ was to identify one particular type of astrophysical source, SN Ia, for a single scientific application, cosmology.
% The original metric was based solely on correctly classifying SN Ia deterministically, with no penalty for confusing SN II for SN Ibc and vice versa, nor was there any notion of a confidence in a classification.
As the \snphotcc\ was only concerned with SN Ia cosmology, it was effectively binary, in that the metric did not distinguish between non-Ia classes.
The \snphotcc\ metric $\mathcal{FOM} \equiv \epsilon \cdot \tilde{\pi}$,
% \begin{eqnarray}
%   \label{eq:snphotccfom}
%   \mathcal{FOM} &\equiv& \epsilon \times \tilde{\pi}
% \end{eqnarray}
is the product of the efficiency
% \begin{eqnarray}
%   \label{eq:efficiency}
%   \epsilon_{Ia} &=& \frac{N_{Ia}^{\mathrm{true}}}{N_{Ia}^{TOT}}
% \end{eqnarray}
of SN Ia classification and a modification $\tilde{\pi} \equiv \mathrm{TP} / (\mathrm{TP} + r \mathrm{FP})$ of the purity in terms of a penalty factor $r$
% \begin{eqnarray}
%   \label{eq:pseudopurity}
%   \tilde{P}_{Ia} &=& \frac{N_{Ia}^{\mathrm{true}}}{N_{Ia}^\mathrm{true} + W_{Ia}^\mathrm{false}N_{Ia}^\mathrm{false}}
% \end{eqnarray}
% with an additive penalty term with weight $W_{Ia}^\mathrm{false}$
motivated by the potential impact on cosmological parameter constraints due to contamination of the SN Ia sample by non-Ia classes.
The pseudo-purity can be interpreted as the traditional purity when $r = 1$.
The \snphotcc\ chose $r = 3$ based on a conservative limit on the acceptable amount of spectroscopic follow-up resources to waste on non-Ia targets.
% on was related to the size of the spectroscopic subsample as roughly $w = 1 + \epsilon_{spec}^{-1} \gg 1$ but the conservative limit of $W_{Ia}^\mathrm{false} = 3$ was chosen.
% to penalize wasted spectroscopic time over rejected SNe.

% \subsection{Modified deterministic metrics}
% \label{sec:auc}

% \aim{TODO: write about AUC so it can be included in results since Michelle calculated it for us!}
% While commonly-used metrics like Receiver Operator Characteristic (ROC) curve (which maps the true positive rate against the false positive rate) and the area under the curve (AUC) can be used to infer probabilities from such classifiers, the fact that \plasticc\ is by construction simulating multiple types of objects makes the extension of ROC as a metric infeasible.

\subsection{Probabilistic metrics}
\label{sec:probabilistic}

In contrast to \snphotcc's single goal of optimal deterministic classification of a single class, \plasticc\ seeks to identify classifiers that produce classification posteriors.
We consider two metrics of classification probabilities that avoid reducing probabilities to deterministic labels.
These metrics are somewhat more involved than those of Section~\ref{sec:deterministic}, requiring additional infrastructure.

% \aim{TODO: will need to fix indexing at 0 in plots!}
Our probabilistic metrics are composed of quantities defined for each possible class $m$ available to light curve $n$, which is a true member of the set $\mathbb{S}_{m'}$ of astrophysical sources of class $m'$.
The metric value $Q_{n} = \sum_{m=1}^{M} Q_{n, m}$ for a single light curve $n$ is a sum of the per-class per-light curve metric values $Q_{n, m}$.
The metric value $Q_{m'} = \sum_{n \in \mathbb{S}_{m'}} Q_{n}$ for an entire class $m'$ is the sum of the per-light curve metrics.
Section~\ref{sec:weights} discusses how the global metrics are derived from the per-class metrics $Q_{m'}$.
As part of the derivation of the per-class per-light curve metrics, we also define the indicator variable
\begin{eqnarray}
  \label{eq:indicator}
  \tau_{n, m} &\equiv& \begin{cases}
  0 & m' \neq m\\
  1 & m' = m
  \end{cases}.
\end{eqnarray}

\subsubsection{Log-loss}
\label{sec:logloss}

The log-loss is a quantity borrowed from information theory and is related to a notion of \textit{entropy} $H_{n} = - \sum_{m=1}^{M} p(m \mid d_{n}) \ln[p(m \mid d_{n})]$, a measure of the space of possible states a system can have, which is in this case the class of which a light curve can be a member.
A classification posterior $p(m \mid d_{n})$ has minimal entropy if it takes a value of $1$ at some class and values of $0$ at all others, i.e. if it can trivially be reduced to a deterministic classification, because this is the scenario in which there is only one possible state, that the light curve has a true class $m$.
This definition of entropy, however, is a property of the probability $p(m \mid d_{n})$ and has no relation with any concept of the true class of the light curve $m'$.

To reconcile the classification posterior with the true class known by those running a challenge, we define the cross-entropy
\begin{eqnarray}
  \label{eq:logloss}
  L_{n} &=& -\sum_{m=1}^{M} \tau_{n, m} \ln[p(m \mid d_{n})],
\end{eqnarray}
which can be interpreted as the spuriously oversized space of possible states (an increase in disorder) due to using the classification posterior in place of the indicator variable.
Whereas $H_{n}$ is minimized to a value of $0$ by any deterministic classification, $L_{n}$ is minimized to a value of $0$ only if $\tau_{n}$ and $p(m \mid d_{n})$ are equal to one another.
It can also be proven that the uncertain classifier of Section~\ref{sec:uncertaindata} maximizes $L_{n}$ \citep{murphy_machine_2012}.
As an aside, a difference between $L_{n}$ and $H_{n}$ evaluated at $\tau_{n, m}$ would be the information lost to disorder in using $p(m \mid d_{n})$ in place of $\tau_{n, m}$, also known as the Kullback-Leibler Divergence (KLD); see \citet{malz_approximating_2018} for a comprehensive exploration of the KLD for a continuous 1-dimensional probability space.

The log-loss has only recently established a presence in the astronomy literature \citep{hon_deep_2017, hon_deep_2018}.
Its greatest strength is that it is straightforwardly interpretable, enabling the metric itself to contribute to uncertainty propagation in an inference problem using the probability densities provided by the classifier.
% \aim{[Rahul: Are you saying you could rewrite a BEAMs like model in terms of the logloss metric rather than the Probabilities P(m|d)?]}
% \aim{[Alex: Not rather than, but as the performance metric of how it's doing.  When propagated through a cosmology calculation, we'd be able to say that BEAMS improves the cosmological parameters by preserving X more information than the alternative.]}

\subsubsection{Brier score}
\label{sec:brier}

The Brier score \citep{brier_verification_1950}, given as
\begin{eqnarray}
  \label{eq:brier}
  B_{n} &=& \sum_{m=1}^{M} (\tau_{n, m} - p(m \mid d_{n}))^{2},
\end{eqnarray}
is a mean square error calculated between the indicator variable and the classification posterior.
Unlike the log-loss, the Brier score has been used extensively in solar flare forecasting \citep{crown_validation_2012, mays_ensemble_2015, florios_forecasting_2018}, stellar variability identification \citep{richards_construction_2012, armstrong_k2_2016}, and star-galaxy separation \citep{kim_hybrid_2015}.

As with the log-loss, the Brier score is minimized to $0$ only for a perfect classifier.
The Brier score is an attractive option because it both rewards classifiers for  assigning more probability to the true class and penalizes classifiers for assigning any probability to classes other than the true class, in contrast to the log-loss, which only accounts for probability assigned to the true class.
We expect this difference to significantly distinguish the Brier score from the log-loss.

The interpretation of the Brier score is less obvious than that of the log-loss, as its dimensions depend on those of the probability space upon which the classification posteriors are defined.
In addition, modifying it with weights requires choosing whether to weight only per-object values $B_{n}$ or also the individual terms $B_{n, m}$ contributing to it.
We leave to future work the thorough investigation of a nontrivial weighting scheme on the Brier metric, however, opting to treat both metrics the same, according to the weighting scheme of Section~\ref{sec:weights}, in our implementation.

\subsection{Weights}
\label{sec:weights}
% TODO: motivate weights as opposed to flat
% TODO: rename section

The most concerning systematics discussed in Section~\ref{sec:mockdata} are those of tunnel vision and cruise control.
The actual light curve data stream of \lsst\ will be particularly vulnerable to both due to extreme class imbalance and hierarchy punctuated by nonrepresentative training data, so \plasticc\ is designed to include these effects.
Any metric under equal weight per light curve would incentivize tunnel vision and cruise control focused on the most prevalent class.
In order to meet the needs of science cases concerning other, rarer classes, \plasticc's metric will be more nuanced, even if it complicates the interpretability of the metric.
% This can be problematic because our science needs might require accurate classifications of uncommon classes as well.
% Because tunnel vision is actually strong performance, it is common for tunnel classifiers to dominate challenges.
% Under a ``winner takes all'' challenge and with equal weight per light curve, \plasticc\ would be particularly vulnerable to a tunnel vision classifier winning despite not meeting the needs of those studying less common classes.

One option is to apply a threshold of classification efficacy on all classes in order to assign an overall winner, though it would require reducing the classification probabilities to deterministic class labels.
When doing binary classification with a method that reduces probabilities to deterministic class labels, each light curve is assigned the class of higher probability, even if the two probabilities are quite similar, a situation that is particularly likely if the light curve, in fact, belongs to a third class or if the two classes are subclasses of a single physical phenomenon.
%We thus anticipate it to be a more severe issue for \plasticc\ and other realistic multi-class challenges as well as any challenges with multiple subclasses.
A simple reduction to a deterministic label could be made more palatable with a secondary threshold mechanism.
For example, requiring a minimum difference in probability density between the maximum probability class and the next highest probability class would help avert this degeneracy.
% (e.g. a newly discovered supernova with a very small number of points may be indistinguishable from a Cataclysmic variable going through a brightening).

A simpler alternative that we investigate in this paper is to use a weighted average
\begin{eqnarray}
  \label{eq:weightavg}
  Q_{m} &=& \frac{1}{\sum_{n} w_{n}} \sum_{n=1}^{N} w_{n} \sum_{m=1}^{M} Q_{n, m}
\end{eqnarray}
of per-class metrics.
(While weights could be assigned to each term $Q_{n, m}$, we do not consider this complexity at this time.)
Weights that are not proportional to $N^{-1}$ nor $M^{-1}$ may be chosen to encourage challenge participants to direct more attention to classes with less active classification efforts or those that have been historically more difficult to classify due to observational limitations.

Downweighting the metrics of classes affected by counterproductive systematics could mitigate the impact of the tunnel vision or cruise control classifiers.
The weights for the \plasticc\ metric, however, must be determined before there is knowledge of which systematics affect which classes.
Because of this caveat, the choice of weights is isolated to an inherently human problem dictated by the value placed on the scientific merits of knowledge of each class.
This paper, on the other hand, can only quantify the impact of weights in relation to the systematics.
We thus agnostically test weighting schemes where classes affected by a particular systematic take a given weight $0 \leq w \leq 1$ and all other classes have a weight $(1 - w) / (M - 1)$.
