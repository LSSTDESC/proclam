\section{Introduction}
\label{sec:intro}

The Large Synoptic Survey Telescope (\lsst) has the potential to advance time-domain astronomy, with anticipated impacts on the study of transient and variable objects within and beyond the Milky Way.
% Bright cosmological objects like Type Ia supernova are probes of cosmological distance (and thus the expansion of the universe).
% Core-collapse supernovae encode within their stunning demise the evolution properties of massive stars.
% Bright astrophysical transients like RR Lyrae give insight into the structure and evolution of stars.
% Active galactic nuclei probe the evolution of large massive galaxies.
% These are but some of the physical principles that are testable with the many different kinds of transients that will be delivered with the Large Synoptic Survey Telescope (LSST).
With its rapid scan strategy, exquisite depth, and many photometric filters, \lsst\ will deliver millions of time-series observations in multiple electromagnetic wavelength ranges (bands, or filters), centered on specific wavelengths of light. These are called `lightcurves' in astronomy, enabling unprecedented population-level studies of time-varying astrophysical sources.
% , in some cases across cosmic time.

Science output from the \lsst\ dataset is contingent on distinguishing classes of astrophysical sources from one another.
From obtaining population statistics of variable stars in the Milky Way to constraining the cosmological parameters with supernovae to discovering optical counterparts to multi-messenger events, accurate classifications are necessary to take advantage of the \lsst\ data volume.
The gold standard for such identification in astronomy has traditionally been based on the spectrum of the source.
However, the volume of objects anticipated of LSST, as well as potentially low signal-to-noise ratios for most sources, likely exceeds the availability of spectroscopic follow-up resources; the great majority of \lsst's time-varying discoveries will never be spectroscopically confirmed.
% Thus several science cases (such as SN cosmology) will actively depend om classification of astrophysical sources based on the photometric light curve, and possibly a much smaller training sample/model based on a spectroscopic sub-sample.
As such, there is an acute need for classifiers of photometric lightcurves that can perform well on datasets that include a wide variety of sources, and where classification over the range of objects is desired, rather than challenges that focus on only one class.

The Photometric \lsst\ Astronomical Time-series Classification Challenge (\plasticc) aims to identify and motivate the development of classification techniques that serve the astronomical community by engaging the broader community outside astronomy.
\plasticc's dataset is comprehensive and intended to encompass all the observations \lsst\ will provide in the coming years.
The simulation includes models for well-understood classes, newly observed classes, and classes that have only been proposed to exist, to simulate serendipitous discoveries anticipated of \lsst.
Additionally, \plasticc\ will join the ranks of a handful\footnote{https://www.kaggle.com/c/DarkWorlds, https://www.kaggle.com/c/galaxy-zoo-the-galaxy-challenge} of past astronomy classification challenges hosted on Kaggle\footnote{https://www.kaggle.com/competitions}, a platform for predictive modeling, that hosts data analytics competitions where seasoned professionals and amateurs alike can compete to classify, model, and predict large data sets uploaded by companies or scientific collaborations.
Kaggle attracts a broad user base, and those without domain knowledge may provide novel approaches to the problem at hand.

Classification in astronomy is increasingly done based on images e.g. galaxy classification \citep{2016A&C....16...34H}, supernova classification \citep{2017ApJ...836...97C}, identifying bars in galaxies \citep{2018MNRAS.477..894A}, separating Near Earth Asteroids from artifacts in images \citep{2016PASJ...68..104M}, as well as light curves e.g. \citet{2016PASJ...68..104M, 2017arXiv170906257M, 2017CQGra..34f4003Z}, and even noise classification \citet{2017CQGra..34f4003Z, 2018PhRvD..97j1501G}.
Automated classification \cite{2011arXiv1110.4655D, 2012arXiv1209.1681D, 2018ApJS..236....9N,2012PASP..124.1175B} is becoming increasingly important in astronomy as the sooner one can make follow-up observations of an interesting object, the more one can learn about its underlying physical processes and nature.

Classifaction is intrinsically \textit{probabilistic} in that the goal is to \textit{constrain} the class \textit{conditioned} on limited data, thereby defining a \textit{posterior probability density}, or \textit{classification posterior} for short, over possible classes for each classified object.
Probabilities of classification that are reduced to point estimates of class become \textit{deterministic} classifications.
As an example, an object with probability $0.9$ of being in a certain class $m$ might be assigned a deterministic classification of being of class $m$, effectively rounding up the probability to $1$, yielding a point estimate of $m$ for that object.
Such a reduction of a probability density to a point estimate discards information whose impact depends on how the classification results are subsequently used.

For example, the decision of how to allocate limited spectroscopic follow-up resources could benefit from probabilistic information if provided early enough.
To reduce wasting spectroscopic resources dedicated to common class whose science use requires spectra, one might only attempt follow-up observations of the objects with the highest classification probabilities.
Spectroscopic follow-up of a rare class, on the other hand, may merit less conservative allocation; an object with even a moderate probability of being of a very rare class could be worth the risk.
% While the knowlege that an object has been identified to be a particular type with an overwhelming probability is likely different from a situation where an object is simply slightly more likely to be a particular class than the remaining classes of objects is important but could be appropriately reflected in a deterministic classifier.
% If follow-up resources were sufficiently abundant to warrant the optimization of follow-up of less well classified objects, the information in the probabilities could be useful in resource allocation.

Classification probabilities can also be propagated through a hierarchical inference of population-level parameters, enabling scientific investigations to proceed even when spectra are unavailable.
The efficacy of this application of classification probabilities has been demonstrated in the context of supernova cosmology \citep{roberts_zbeams:_2017}.
% TODO Cite all the BEAMS papers and Malz, Peters, Hlozek in prep
% Considering the different case of supernova cosmology from a photometrically classified sample of SNe, such probabilistic classifications can be propagated to subsequent analyses of cosmological parameters allowing one to extract information from the (large) part of the sample where photometric classification failed to identify a particular type with overwhelming probability \cite{roberts_zbeams:_2017}.
Thus the impact of a photometric-only survey like \lsst\ can be greatly enhanced by probabilistic classifications.
%Because of the low-signal-to-noise expected of \lsst, probabilistic classifications are more appropriate (Roberts+17) than the point estimates of previous challenges.
%Such posteriors are more valuable than point estimates, which we call deterministic classifications in this work, because of their versatility in application and encapsulation of observational and systematic error that may propagate through inference.

% A classification posterior can be propagated through population-level inference without a separate error propagation pipeline, and a probability can always be reduced to a point estimate for the purposes of decision-making (such as for allocation of follow-up resources), but a deterministic classification cannot in general be used to reconstruct a probability density for an individual object.
In light of the aforementioned benefits of classification probabilities, \plasticc\ will thus accept classifiers producing classification posteriors.\footnote{Many classifiers, including some of the most prevalent classifiers in the time-domain astronomy field, that only provide deterministic and/or binary classifications will have to convert their results to probability vectors to compete in \plasticc.}

However, probabilistic classifications are incompatible with the metrics of deterministic class assignments used in previous classification challenges \citep{kessler_supernova_2010, kessler_results_2010} and efforts to develop supernova classifiers \citep{2018ApJS..236....9N}.
In this context, a metric is simply a quantification of the performance of a classifier.
Notions of accuracy, purity, completeness, and other metrics endemic to science are examples of deterministic metrics appropriate to classification point estimates.
Many deterministic classification metrics can be modified for evaluation on probabilistic classifications \citep{lochner_photometric_2016, moller_photometric_2016, hon_deep_2017, hon_detecting_2018, 2011arXiv1108.4696G} but only by reducing class probabilities to point estimates of class and evaluating those at discrete cutoffs, the choice of which may ultimately affect the conclusions of the study.

If the data are simulated using a fully self-consistent forward model, a metric of the accuracy of classification probabilities relative to the true, underlying probabilities would be straightforward.
However, such a simulation procedure would require beginning with a fully populated probability space over all classes and all possible lightcurves, which is an insurmountable challenge.
Therefore, attention must be directed toward the no longer straightforward matter of defining the criterion for a winning classifier.
In the context of astronomy, concerns about the choice of metric for probabilistic classifications have been investigated only to a limited degree thus far \citep{2018SoPh..293...28F, 2017MNRAS.464.4463K}
% TODO cite PSNID
, with most approaches concentrating on the standard metrics of purity and completeness, however metric consistency over a range of classifiers and between different analyses is not always ensured \citep{2018A&C....23...15B}.

This work explores the problem of choosing a metric of probabilistic classifications, which will be used in many science applications and thus lack a single obvious figure of merit.
The winning classifier should be robust to significant class imbalance, both between the training set and test set and within either, and other concerning systematics.

The \plasticc\ metric must be well-defined for non-binary classes, going beyond the positive/negative dichotomy inherent to some traditional metrics.
It must also respect the information content of probabilistic classifications without reduction to point estimates of class.
%\aim{I actually preferred the following as a list, although I agree that it could use some pruning.}
%\begin{itemize}
%\item    The metric must return a single scalar value.
%\item    The metric must be well-defined for non-binary classes.
%\item    The metric must balance diverse science use cases in the presence of heavily nonuniform class prevalence.
%\item    The metric must respect the information content of probabilistic classifications.
%\item    The metric must be able to evaluate deterministic classifications.
%\item    The metric must be interpretable, meaning it gives a more optimal value for "good" mock classifiers and a less optimal value for mock classifiers plagued by anticipated systematic errors; in other words, it must pass basic tests of intuition.
%\item    The metric must be reliable, giving consistent results for different instantiations of the same test case.
%\end{itemize}
In order for the metric to satisfy the challenge requirements, the metric must return a single, scalar value.
It also must be reliable, giving consistent results for different instantiations of the same test case.
% In addition, the metric should balance diverse science use case needs, particularly in the presence of heavily non-uniform classes.
% Deterministic classifications must be converted to probabilistic (i.e. returning a ``1'' or a ``0''), that deterministic metric must be readily convertible into a probabilistic classification - and should preserve the relationships between classes accordingly.
% Similarly, any probabilistic metric much be easily transferred to a deterministic one ( e.g. given a threshold on the most likely classficiation choice).

We perform a systematic exploration of the sensitivity of metrics of probabilistic classification to anticipated classifier failure modes using the Probabilistic Classification Metric (\proclam) code, which is is publicly available on GitHub.\footnote{\url{https://github.com/aimalz/proclam}}.
The mock classification results that we use for this exploration are described in Section~\ref{sec:data}.
The metrics we consider are presented in Section~\ref{sec:methods}.
The behavior of the metrics as a function of mock classification results is presented in Section~\ref{sec:results}.
We discuss extensions of this exploratory framework to more complex challenge goals in Section~\ref{sec:discussion}.
